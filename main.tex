% !TeX spellcheck = es_PE
% document configuration
\include{document_configuration}

% introduction parameters
\title[Laboratorio 0]{Control no lineal}
\subtitle{\textbf{Introducción a los módulos de control y Simulink}}

\author[J. Charaja and R. Terreros (UTEC)]{\textbf {Profesora} \\ Ruth Canahuire \\ \vspace{1em} \textbf{Asistentes de enseñanza} \\ Jhon Charaja y Ricardo Terreros}

\logo{\includegraphics[height=8em]{imgs/utec-logo.png}}

\date{\today}


\begin{document}
	\frame{\titlepage}
	% primer slide
	% describir los componentes del módulo de control (QUBE-Servo 2) y cómo lo vamos a usar
	% una foto de la placa de conexiones, una foto de cubo (enfocar conexión), una foto del péndulo y una foto del disco de masa
	
	% segundo slide:
	% mostrar como conectar el módulo, imágen de las conexiones (placa -> módulo) y una imágen del cubo con el péndulo y disco de masa
	
	% tercer slide:
	% indicar cúales son los bloques de Quanser para recibir data (posición) y enviar data (señal de control)
	
	% cuarto slide
	% mostar el diagrama de bloques de control de posición en Simulink .
	
	% quinto slide
	% controlar velocidad
	
	\section{Introducción}

	\begin{frame}
		\frametitle{Módulos del laboratorio L414}
		\begin{minipage}[t]{0.47\textwidth}
			\graphicspath{{imgs/qube/}}
			\centering
			\animategraphics[controls={play}, loop, width=\textwidth]{60}{}{1}{450} \\
			\vspace{1em}
			{\large \textbf{QUBE-Servo 2\footnotemark[1]}}
		\end{minipage}
				
		\footnotetext[1]{Video recuperado de \url{https://www.quanser.com/products/qube-servo-2/} }
	\end{frame}
\end{document}