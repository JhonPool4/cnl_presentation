% !TeX spellcheck = es_PE
% document configuration
\include{document_configuration}

% introduction parameters
\title[Laboratorio 0]{Control no lineal}
\subtitle{\textbf{Introducción a los módulos de control y Simulink}}

\author[J. Charaja and R. Terreros (UTEC)]{\textbf {Profesora} \\ Ruth Canahuire \\ \vspace{1em} \textbf{Asistentes de enseñanza} \\ Jhon Charaja y Ricardo Terreros}

\logo{\includegraphics[height=8em]{imgs/utec-logo.png}}

\date{\today}


\begin{document}
	\frame{\titlepage}
	\section{Introducción}
	% hablar sobre los módulos que vamos a usar	
	\begin{frame}
		\frametitle{Módulos del laboratorio L414}
		
		\begin{minipage}[t]{0.47\textwidth}
			\graphicspath{{imgs/qube/}}
			\centering
			\animategraphics[controls={play}, loop, width=\textwidth]{60}{}{10}{20} \\
			\vspace{1em}
			{\large \textbf{QUBE-Servo 2\footnotemark[1]}}
		\end{minipage}
		%\hspace{.08\textwidth}
		\begin{minipage}[t]{0.47\textwidth}
			\graphicspath{{imgs/inverted_pendulum/}}
			\centering
			\animategraphics[controls={play}, loop, width=\textwidth]{60}{}{10}{20} \\
			\vspace{1em}
			{\large \textbf{Inverted Pendulum\footnotemark[2]}}
		\end{minipage}
				
		\footnotetext[1]{Video recuperado de \url{https://www.quanser.com/products/qube-servo-2/} }
		\footnotetext[2]{Video recuperado de \url{https://www.quanser.com/products/qube-servo-2/} }
	\end{frame}
\end{document}